% Opcje klasy 'iithesis' opisane sa w komentarzach w pliku klasy. Za ich pomoca
% ustawia sie przede wszystkim jezyk i rodzaj (lic/inz/mgr) pracy, oraz czy na
% drugiej stronie pracy ma byc skladany wzor oswiadczenia o autorskim wykonaniu.
\documentclass[declaration,shortabstract,mgr]{settings/iithesis}

\usepackage[utf8]{inputenc}
\usepackage{multirow}
\usepackage{makecell}
\usepackage{graphicx}
\usepackage{amsthm}
\usepackage{amsfonts} 
\usepackage{subcaption}
\usepackage{amsmath}
\usepackage{tikz}
\usepackage[ruled]{algorithm2e}
\usepackage{listings}
\usepackage{float}
\usepackage{array}

\usetikzlibrary{arrows}

\tikzset{
  treenode/.style = { circle, draw, minimum size=0.75cm, align=center, inner sep=0pt, text centered,
    font=\sffamily},
}

\theoremstyle{definition}
\newtheorem{theo}{Twierdzenie}[chapter]
\newtheorem{ex}{Ćwiczenie}[chapter]
\newtheorem{example}{Przykład}[chapter]
\newtheorem{fact}{Fakt}[chapter]
\newtheorem{definition}{Definicja}[chapter]
\newtheorem{axiom}{Aksjomat}[chapter]

%%remove double dot 
\renewcommand{\thetheo}{\arabic{theo}}
\renewcommand{\theexample}{\arabic{example}}
\renewcommand{\theex}{\arabic{ex}}
\renewcommand{\thefact}{\arabic{fact}}
\renewcommand{\thedefinition}{\arabic{definition}}
\renewcommand{\theaxiom}{\arabic{axiom}}

%% algorithm2e translations:

\renewcommand{\algorithmcfname}{Algorytm}

\newcommand{\set}[2]{ \{\ #1\ |\ #2\ \} }
\newcommand{\pair}[2]{\langle #1,#2 \rangle }
\newcommand{\tuple}[1]{\langle #1 \rangle}
\newcommand{\Xleftrightarrow}[1]{\stackrel{#1}{\Longleftrightarrow}}
\newcommand{\xleftrightarrow}[1]{\stackrel{#1}{\longleftrightarrow}}
\newcommand{\xequiv}[1]{\stackrel{#1}{\equiv}}
\newcommand{\m}[1]{\({ #1 }\)}
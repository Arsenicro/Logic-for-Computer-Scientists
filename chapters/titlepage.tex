%%%%% DANE DO STRONY TYTUŁOWEJ
% Niezaleznie od jezyka pracy wybranego w opcjach klasy, tytul i streszczenie
% pracy nalezy podac zarowno w jezyku polskim, jak i angielskim.
% Pamietaj o madrym (zgodnym z logicznym rozbiorem zdania oraz estetyka) recznym
% zlamaniu wierszy w temacie pracy, zwlaszcza tego w jezyku pracy. Uzyj do tego
% polecenia \fmlinebreak.
\polishtitle    {Jak upolować żywego lwa? Logika dla informatyków}
\englishtitle   {How to hunt a live lion? Logic for Computer Scientists}
\polishabstract {,,Logika dla Informatyków'' jest jednym z pierwszych przedmiotów obowiązkowych z którymi musi sobie poradzić każdy student informatyki na Uniwersytecie Wrocławskim. Mimo, że przedmiot nie jest trudny, to od wielu lat sprawia problemy studentom. Przez to, że logika znacznie różni się od poznanej w szkole średniej matematyki, niektórzy studenci mają problemy z opanowaniem początkowych tematów, a niezrozumienie poprzednich tematów jeszcze bardziej utrudnia zrozumienie dalszego materiału. Składająca się z części pisemnej i części audiowizualnej praca stanowi próbę stworzenia poradnika, który mógłby pomóc studentom pierwszego roku lepiej poradzić sobie z logiką i w efekcie mieć większe szanse by ukończyć studia na Uniwersytecie Wrocławskim.}
\englishabstract{,,Logic for Computer Scientists'' is one of the first mandatory classes that every computer science student at the University of Wrocław has to deal with. Although the subject is not difficult, it has been causing problems for students for many years. The fact that for many students logic is something completely different from the math they have learned in high school makes it difficult for them to master the starting material, and not understanding the first few topics makes it very difficult to understand the rest of the material. Consisting of a written part and an audiovisual part, the work is an attempt to create a guide that could help first-year students better cope with logic and, as a result, have a better chance of graduating from the University of Wrocław.}
% w pracach wielu autorow nazwiska mozna oddzielic poleceniem \and
\author         {Kamil Matuszewski}
% w przypadku kilku promotorow, lub koniecznosci podania ich afiliacji, linie
% w ponizszym poleceniu mozna zlamac poleceniem \fmlinebreak
\advisor        {dr Piotr Wieczorek}
%\date          {}                     % Data zlozenia pracy
% Dane do oswiadczenia o autorskim wykonaniu
%\transcriptnum {}                     % Numer indeksu
%\advisorgen    {dr. Jana Kowalskiego} % Nazwisko promotora w dopelniaczu
%%%%%
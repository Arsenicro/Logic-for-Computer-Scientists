\section{Wstęp}
Przedmiot ,,Logika dla Informatyków'' jest przedmiotem obowiązkowym, z którym każdy student Informatyki na Uniwersytecie Wrocławskim musi zdać już na pierwszym semestrze studiów. Choć z perspektywy czasu jest to przedmiot dość prosty, to co roku pojawiają się studenci, którzy nie potrafią sobie z nim samodzielnie poradzić. Wynika to głównie z tego, że jego tematyka różni się znacząco od treści nauczanych w szkołach średnich.

Poniższa praca jest szkicem podręcznika, który każdy student pierwszego roku mógłby przeczytać, w celu lepszego zrozumienia poszczególnych działów przedmiotu ,,Logika dla Informatyków''. Podręcznik ten miałby być przystępny dla osób ze znajomością matematyki na poziomie szkoły średniej, jednak nie posiadających wiedzy z zakresu logiki matematycznej. Jako, że stworzenie obszernego podręcznika jest zadaniem wykraczającym poza wymagania pracy magisterskiej, praca ta jest jedynie koncepcją, przedstawiającą kilka wybranych zagadnień.

Praca powstała w oparciu o materiały ze skryptu do przedmiotu ,,Logkia dla Infomratyków'' \cite{skrypt}, a także skrypt ,,Wstęp do teorii mnogości i logiki'' \cite{tiuryn}.

Oprócz części pisemnej, do pracy została załączona część audiowizualna, na której autor pracy, w formie ,,wykładu online'' opowiada o wybranych zagadnieniach. 

\section{Część pisemna}
Część pisemna pracy pisana była w oparciu o ,,Materiały do zajęć'' i składa się z czterech tematów.

\begin{enumerate}
    \item \hyperref[chapter:proofs]{,,Techniki dowodowe''} jest rozdziałem wprowadzającym, w którym czytelnik zapozna się z pojęciem dowodu. Przedstawione w nim zostaną, na przykładzie kilku prostych twierdzeń, popularne techniki dowodowe, dzięki którym czytelnik lepiej zrozumie czym jest dowód i jak powinno się przedstawiać tok rozumowania. Dzięki temu kolejne rozdziały, a także tematyka samego przedmiotu, powinny być bardziej przystępne.
    \item \hyperref[chapter:induction]{,,Indukcja Matematyczna''} jest rozdziałem opisującym pierwszy temat, z którym spotykają się studenci na przedmiocie ,,Logika dla Informatyków''. W tym rozdziale opisana jest indukcja matematyczna, wraz z podstawowymi definicjami a także przykładami użycia, dzięki którym czytelnik lepiej zrozumie czym jest indukcja matematyczna oraz jak powinno się jej używać.
    \item Kolejny rozdział \hyperref[chapter:sets]{,,Zbiory''} zaczyna się od przedstawienia podstawowych definicji zbioru oraz operacji na zbiorach, następnie omawia tematykę rodzin zbiorów, na koniec skupiając się na zbiorach nieskończonych i operacjach na zbiorach nieskończonych. W tym rozdziale znajdują się także przykłady zadań, które pomogą zrozumieć definicje i ich zastosowanie.
    \item W rozdziale \hyperref[chapter:relation]{,,Relacje''} zawarte są definicje pary, relacji, operacji na relacjach a także relacyjny rachunek dziedzin, oraz przykładowe zadania związane z tymi definicjami.
    \item Ostatni rozdział, \hyperref[chapter:ending]{,,Zakończenie''} oprócz podsumowania pracy, przedstawia również jak można zastosować niektóre z przedstawionych wyżej tematów w praktyce. Pierwszy przykład pokaże, jak można użyć rozumowania indukcyjnego do udowodnienia czegoś o pewnym popularnym algorytmie, natomiast drugi przykład pokazuje, jak można przełożyć zapytanie w relacyjnym rachunku dziedzin na zapytanie w języku SQL.
\end{enumerate}

\section{Część audiowizualna}
W skład części audiowizualnej wchodzi 28 filmików, o łącznej długości około 4 godzin. Na tych materiałach autor pracy opowiada o wybranych zagadnieniach przedmiotu ,,Logika dla Informatyków''.

Część audiowizualna jest niezależna od części pisemnej: nie trzeba zapoznawać się z jedną, żeby zrozumieć drugą.

Materiały te stanowiły część projektu, w ramach którego na kanale ,,Logika dla Informatyków UWr'' na portalu YouTube co tydzień przez 9 tygodni publikowane było omówienie materiału z danego tygodnia. Materiały, tworzone przez Kamila Matuszewskiego, Bartosza Bednarczyka oraz Annę Karykowską, udostępniane były następnie studentom. 

Materiały czasem mogą wspominać części które tworzone były przez współautorów kanału ,,Logika dla informatyków UWr'', zakładają też zrozumienie poprzednich materiałów w celu zrozumienia kolejnych.

Poniżej znajduje się spis materiałów, w kolejności, w której były one publikowane. W nawiasach dopisane zostały numery wykładów, oraz numer części którą dany materiał stanowił, jako część wszystkich materiałów opublikowanych w danym tygodniu.

\begin{itemize}
    \item ,,Dowody Niekonstruktywne'' (Wykład 0, część 3)
    \item ,,Silniejsza indukcja'' 1, 2 i 3 (Wykład 1, część 5, 6 i 7)
    \item ,,Silna indukcja'' 1 i 2 (Wykład 1, część 8 i 9)
    \item ,,Wstęp do rachunku zdań'' 1 i 2 (Wykład 2, część 1 i 2)
    \item ,,Wartościowanie zmiennych i wartościowanie formuł'' (Wykład 2, część 3)
    \item ,,Negacyjna postać normalna'' 1 i 2 (Wykład 3, część 1 i 2)
    \item ,,Funkcje boolowskie'' (Wykład 4, część 1)
    \item ,,Zbiory zupełne spójników: Definicja'' (Wykład 4, część 2)
    \item ,,Rezolucja'' 1, 2 i 3 (Wykład 4, część 5, 6 i 7)
    \item ,,Zbiory'' 1, 2, 3 i 4 (Wykład 6, część 4, 5, 6 i 7)
    \item ,,Zbiory nieskończone'' 1, 2 i 3 (Wykład 7, część 1, 2 i 3)
    \item ,,Relacje'' 3, 4 i 5 (Wykład 7, część 9, 10 i 11)
    \item ,,Funkcje'' 1 i 2 (Wykład 8, część 4 i 5)
\end{itemize}